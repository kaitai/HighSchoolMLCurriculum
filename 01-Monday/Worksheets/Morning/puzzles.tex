\documentclass[10pt]{article}
\usepackage{itcep, stmaryrd, tikz, pgflibraryplotmarks, multicol, pgfplots}
\usepackage[margin=1in, nohead, pdftex]{geometry}

\topmargin -0.2in
\pagestyle{empty}
\singlespacing
\let\oldhat\hat
\renewcommand{\vec}[1]{\mathbf{#1}}
\renewcommand{\hat}[1]{\oldhat{\mathbf{#1}}}

\definecolor{light-gray}{gray}{0.95}
\newcommand{\code}[1]{\colorbox{light-gray}{\texttt{#1}}}

\newcommand{\headerclass}{Machine Learning Camp}
\newcommand{\headersection}{Day 1: Exploring Data}
\newcommand{\headertitle}{A Puzzling Picture}

\def\C{\mathbb{C}}
\def\R{\mathbb{R}}
\parindent 0ex
\begin{document}
%==================================================================================================================================================
\headerclass\xspace \hspace{\stretch{1}} \headersection\\
\begin{center}{ \large \textbf{\headertitle} }\end{center}
%==================================================================================================================================================

Thinking about the puzzle activity as a metaphor for working with data, discuss the following questions.

\begin{enumerate}
\item What are some ways that data can be imperfect?
\vfill
\item What types of events can cause imperfect data?
\vfill
\item Even if data is imperfect, can it still be useful?
\vfill
\item What are some strategies you can use to deal with imperfect data?
\vfill
\end{enumerate}

\end{document}
